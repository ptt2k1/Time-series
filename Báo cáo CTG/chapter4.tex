\chapter{KẾT LUẬN}
Trong phạm vi nội dung của bài tập lớn, một số nội dung mà nhóm chúng em đã đạt được:
\begin{itemize}
    \item Thành công trong việc giới thiệu mô hình dự đoán chuỗi thời gian LSTM và mô hình VARMAX. 
    \item Ứng dụng được vào bài toán dự đoán chất lượng không khí.
\end{itemize}
Với những kết quả đạt được, bài tập lớn có nhiều tiềm năng ứng dụng trong nhiều bài toán khác nhau về chuỗi thời gian. Một số hướng phát triển tiếp theo của đồ án:
\begin{itemize}
    \item Cải thiện độ chính xác của mô hình và ứng dụng vào nhiều lĩnh vực khác nhau, ví dụ: dự báo giá chứng khoán, dự đoán doanh số bán hàng, v.v...
    \item Phát triển thêm các biến thể của LSTM như Gated Recurrent Unit (GRU), Depth Gated RNNs hay Clockwork RNNs
\end{itemize}


\newpage
\begin{center}
   \textbf{\LARGE{Tài liệu tham khảo}}
\end{center}
\addcontentsline{toc}{section}{Tài liệu tham khảo}
\begin{enumerate}
    \item Joos Korstanje, \textit{Advanced Forecasting with Python}.
    \item B. Brockwell and R. Davis, "Time series: Theory and methods", Springer-Verlag, 1987.
    \item S. Hochreiter and J. Schmidhuber, "Long short-term memory", \textit{Neural computation}, vol. 9, pp. 1735-80, 12 1997.
    \item \url{https://viblo.asia/p/machine-learning-thu-lam-nha-thien-van-du-bao-thoi-tiet-djeZ1xYmKWz}
    
    \item \url{https://www.statsmodels.org/dev/examples/notebooks/generated/statespace_varmax}
    
    \item \url{https://vi.wikipedia.org/wiki/Sai_s%E1%BB%91_to%C3%A0n_ph%C6%B0%C6%A1ng_trung_b%C3%ACnh}
\end{enumerate}

