\pagenumbering{arabic}
\setcounter{page}{1}
\chapter{MỞ ĐẦU}
\section{Lời mở đầu}
Con người có thể sống sót trong 30 ngày mà không cần ăn, 3 ngày không uống, nhưng sẽ chết chỉ sau 3 phút nếu không thở. Nhu cầu về không khí của con người cũng không đổi, chúng ta hít thở nó thường xuyên dù trong nhà hay ngoài nhà, thậm chí phải chấp nhận cả khi chất lượng không được như mong đợi. Mùi khó chịu khiến con người nhận ra chất lượng không khí không tốt, nhưng nhiều khi khứu giác lại không thể cảm nhận được những tác nhân gây hại đang tồn tại, như các hóa chất độc hại – chúng ảnh hưởng vô cùng tồi tệ tới sức khỏe.

Ngày này chất lượng không khí tại các thành phố lớn ngày càng đi xuống. Bởi vậy việc đo lường, dự đoán chất lượng không khí tại một khu vực nhất định là vô cùng cần thiết. Nó cho ta biết chỉ số ô nhiễm tại nơi đó là bao nhiêu, nếu chỉ số quá cao sẽ ảnh hưởng rất lớn tới sức khỏe con người. 

\section{Giới thiệu bài toán}
\begin{itemize}
    \item [$-$]Bài toán: Nhóm chúng em sẽ sử dụng mô hình LSTM và mô hình VARMAX để dự đoán chất lượng không khí trong 24h tới dựa vào các thông số thời tiết của các ngày trước đó như: nhiệt độ, độ ẩm, hướng gió, áp suất khí quyển, lượng bụi mịn PM10, PM 2.5, nồng độ khí CO, S02, NO trong không khí,...

    \item [$-$]Phạm vi nghiên cứu: từ ngày 14/4/2015 tới ngày 5/9/2019.
\end{itemize}

\newpage
\section{Phân công công việc trong nhóm}
\begin{table}[h!]
    \centering
    \tabcolsep = 0.1cm
    \begin{tabular}{|c|c|}
        \hline
        \textbf{Sinh viên} & \textbf{Công việc}\\ 
        \hline
        Phạm Thị Hoa & Tìm hiểu code và ứng dụng mô hình LSTM vào bài toán\\
        \hline
        Lê Thanh Thảo & Tìm hiểu các mô hình và làm báo cáo\\
        \hline
        Phạm Thu Trang & Tìm hiểu code và ứng dụng mô hình VARMAX vào bài toán\\ 
        \hline
    \end{tabular}
    \caption{Phân công công việc}
    %\label{tab:my_label}
\end{table}